\documentclass[a4paper, 11pt, twocolumn]{article}
\usepackage[utf8]{inputenc}
\usepackage[left=1.8cm,text={18cm, 25cm},top=2.5cm]{geometry}
\usepackage[czech]{babel}
\usepackage[IL2]{fontenc}
\usepackage[unicode]{hyperref}
\usepackage[dvipsnames]{xcolor}
\usepackage{amssymb}
\usepackage{amsmath}
\usepackage{amsthm}
\usepackage{times}
\theoremstyle{definition}
\newtheorem{definition}{Definice}
\newtheorem{sentence}{Věta}
\renewcommand\labelitemi{$\bullet$}

\title{ITY 2}
\author{Simona Češková (xcesko00)}
\date{2021}

\begin{document}
\begin{titlepage}
\begin{center}
\Huge
\textsc{Fakulta informačních technologií
Vysoké učení technické v~Brně} \\
\vspace{\stretch{0.25}}
\LARGE
Typografie a publikování -- 2. projekt \\
Sazba dokumentů a matematických výrazů \\
\vspace{\stretch{0.4}}
\end{center}
{\Large 2021 \hfill Simona Češková (xcesko00)}
\end{titlepage}

\section*{Úvod} \label{1}
V~této úloze si vyzkoušíme sazbu titulní strany, matematických vzorců, prostředí a dalších textových struktur obvyklých pro technicky zaměřené texty (například rovnice~ \eqref{eq:1}
nebo Definice 1 na straně \pageref{1}). Rovněž si vyzkoušíme používání odkazů \verb!\ref! a \verb!\pageref!. \par
Na titulní straně je využito sázení nadpisu podle optického středu s~využitím zlatého řezu. Tento postup byl
probírán na přednášce. Dále je použito odřádkování se
zadanou relativní velikostí 0.4 em a 0.3 em. \par
V~případě, že budete potřebovat vyjádřit matematickou
konstrukci nebo symbol a nebude se Vám dařit jej nalézt
v~samotném \LaTeX u, doporučuji prostudovat možnosti balíku maker \AmS\--\LaTeX.

\section{Matematický text}
Nejprve se podíváme na sázení matematických symbolů
a výrazů v~plynulém textu včetně sazby definic a vět s~využitím balíku amsthm. Rovněž použijeme poznámku pod
čarou s~použitím příkazu \verb!\footnote!. Někdy je vhodné
použít konstrukci \verb!\mbox{ }!, která říká, že text nemá být
zalomen.
\begin{definition} \label{eq:1}
Rozšířený zásobníkový automat {\it (RZA) je definován jako sedmice tvaru $A = (Q, \Sigma, \Gamma, \delta, q_0, Z_0, F)$, kde:}
\end{definition}
\begin{itemize}
\item Q {\it je konečná množina} vnitřních (řídicích) stavů,
\item $\Sigma$ {\it je konečná} vstupní abeceda,
\item $\Gamma$ {\it je konečná} zásobníková abeceda,
\item $\delta$ {\it je} přechodová funkce $Q\times(\Sigma\cup\verb!{!\epsilon\verb!}!)\times\Gamma^{\ast} \rightarrow 2^{Q \times \Gamma^{\ast}}$,
\item $q0 \in Q$ {\it je} počáteční stav, $Z_0 \in \Gamma$ {\it je} startovací symbol
zásobníku a $F \subseteq Q$ {\it je množina} koncových stavů.
\end{itemize} \par
Nechť $P = (Q , \Sigma , \Gamma , \delta , q_0 , Z_0 , F)$ je rozšířený zásobníkový automat. {\it Konfigurací} nazveme trojici $(q, w, \alpha) \in Q \times \Sigma^{\ast} \times\Gamma^{\ast}$, kde q je aktuální stav vnitřního řízení,
$w$ je dosud nezpracovaná část vstupního řetězce a $\alpha = Z_{i1}Z_{i2}\ldots Z_{ik}$
je obsah zásobníku\footnote{$Z_{i_1}$ je vrchol zásobníku}.

\subsection{Podsekce obsahující větu a odkaz}
\begin{definition}
Řetězec $w$ nad abecedou $\Sigma$ je přijat RZA
{\it A~jestliže $(q_0, w, Z_0) \overset{*}{\underset{A}\vdash} (q_F , \epsilon , \gamma)$ pro nějaké $\gamma\in\Gamma^{\ast}$ a $q_F\in F$. Množinu $L(A) = \verb!{!w \mid w$ je přijat RZA A$\verb!}! \subseteq \Sigma^{\ast}$ nazýváme} jazyk přijímaný RZA {\it A}.
\end{definition}
Nyní si vyzkoušíme sazbu vět a důkazů opět s~použitím
balíku amsthm.
\begin{sentence}
{\it Třída jazyků, které jsou přijímány ZA, odpovídá}
bezkontextovým jazykům.
\end{sentence}
\begin{proof}
V~důkaze vyjdeme z~Definice 1 a 2.
\end{proof}

\section{Rovnice a odkazy}
Složitější matematické formulace sázíme mimo plynulý
text. Lze umístit několik výrazů na jeden řádek, ale pak je
třeba tyto vhodně oddělit, například příkazem \verb!\quad!.
$$\sqrt[i]{x_i^3}\textrm{ kde } x_i \text{ je } i\text{-té sudé číslo splňující } x_i^{x_i^{i^2} +2} \leq y_i^{x_i^4}$$\par
V~rovnici \eqref{eq:2} jsou využity tři typy závorek s~různou
explicitně definovanou velikostí.

\begin{equation} \label{eq:2}
x=\bigg[\Big{\{} \big[a+b \big]\ast c \Big{\}} ^d \oplus 2 \bigg]^{3/2}
\end{equation}
\begin{equation*}
y = \lim_{x\to\infty} \frac{\frac{1}{\log_{10} x}}{\sin^2{x} + \cos^2{x}}
\end{equation*}

V~této větě vidíme, jak vypadá implicitní vysázení limity $\lim_{n\to\infty} f(n)$ v~normálním odstavci textu. Podobně
je to i s~dalšími symboly jako $\prod_{i=1}^n 2^i$
či $\bigcap_{A \in B} A$. V~případě vzorců ${\displaystyle\lim_{n\to\infty} f(n)}$
a
$\overset{n}{\underset{i=1}\prod} 2^i$
jsme si vynutili méně
úspornou sazbu příkazem \verb!\limits!.
$ $
\begin{equation}
\int_{b}^{a} g(x)\,dx = - \int\limits_a^b f(x)\mathrm{d}x  
\end{equation}

\section{Matice}
Pro sázení matic se velmi často používá prostředí \verb!array!
a závorky (\verb!\left!, \verb!\right!).

\begin{equation*}
\left ( {\begin{array}{ccc}
a-b & \widehat{\xi + \omega} & \pi\\
\vec {\rm\bf a} & \overleftrightarrow{AC} & \hat{\beta}
\end{array} } \right )
=
1 \Longleftrightarrow \mathcal{Q} = \mathbb{R}
\end{equation*}

\begin{equation*}
\textbf{A} =
\begin{Vmatrix} 
{\begin{array}{cccc}
a_{11} & a_{12} & \dots & a_{1n}\\
a_{21} & a_{22} & \dots & a_{2n}\\
\vdots & \vdots & \ddots & \vdots\\
a_{m1} & a_{m2} & \dots & a_{mn}\\
\end{array} } 
\end{Vmatrix}
=
\left | {\begin{array}{cc}
t & u\\
v & w
\end{array} } \right |
= tw-uv
\end{equation*}

Prostředí \verb!array! lze úspěšně využít i jinde.

\begin{equation*}
\binom{n}{k} =
\begin{cases}
{\begin{array}{cl}
0 & \text{pro $k < 0$ nebo $k > n$}\\
\frac{n!}{k!(n-k)!} & \text{pro $0 \leq k \leq n$.}
\end{array} } 
\end{cases}
\end{equation*}

\end{document}
