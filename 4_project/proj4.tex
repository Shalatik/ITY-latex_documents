\documentclass[a4paper, 11pt]{article}
\usepackage[utf8]{inputenc}
\usepackage[left=2cm,text={17cm, 24cm},top=3cm]{geometry}
\usepackage[czech]{babel}
\usepackage[T1]{fontenc}
\usepackage[unicode]{hyperref}
\usepackage{times}
\usepackage{csquotes}

\title{ITY4}
\author{Simona Češková (xcesko00)}
\date{April 2021}

\begin{document}
\begin{titlepage}
\begin{center}
\Huge
\textsc{Vysoké učení technické v~Brně\\
\huge{Fakulta informačních technologií}} \\
\vspace{\stretch{0.382}}
\LARGE
Typografie a publikování -- 4. projekt \\
\Huge{Bibliografické citace} \\
\vspace{\stretch{0.618}}
\end{center}
{\Large \today \hfill Simona Češková (xcesko00)}
\end{titlepage}

\section{Typografie}
\subsection{Vznik typografie}
Vývojem nejjednoduších jazyků se postupně začaly vyvýjet i první komunikační prostředky. Pomocí postupného vylepšování různých nástrojů a zbraní z~kamene se začaly objevovat i první známky písma vytesaného do kamene. A~to byl první krok ke vzniku dnešních tiskovin, tak jak je známe dnes viz \cite{Hamilton2009}.

\subsection{Zásady typografie}
Typografie se řídí všeobecnými zásadami, které je vhodné dodržovat jak z~logického, tak i z~estetického hlediska píše \cite{Sirucek2006}. Právě estetickou a grafickou úpravu můžeme vidět převážně u~časopisů, které se tímto znakem liší od novin a jiných publikací, napříkad novin. \cite{Bartos2017}

\par
První tyto zásady začínají vznikat už od vynálezu tisku, například výběr správného typu písma a jeho kombinace. Dále o~nich pojednává \cite{Sirucek2006}.

\subsubsection{Hladká vazba}
Mezi další zásady typografie patří sazba hladké vazby. K~této sazbě řadíme sazbu pomlček, spojovníků, interpukčních znamének a specilních znaků. Časté chyby při použité hladké sazby dále popisuje \cite{Glac2006}.
\par
Jedna z~chyb, které se můžeme dopustit například je, že před interpunkční znaménko přidáme mezeru, která ho bude spojovat s~předchozím slovem. Avšak správný český zápis je bez této mezery viz \cite{Cmejrkova1999}

\subsubsection{Psaní uvozovek}
Uvozovky slouží k~tomu, aby oddělily ve větě přímouřeč, citáty, názvy a podobně. Správné použití uvozovek je napsat je přímo k~textu na který navazují a to bez mezer viz \cite{Nezbeda2018}.
\par
Tato část typografie je pro řadu lidí stále nejasná a často si správné české psaní uvozovek pletou s~typografií psaní uvozovek v~anglicky mluvících zemích. O~tom, jak tomuto omylu zamezit při použítí textového editoru, dále pojednává \cite{Konecny2015}.

Avšak mohou se vyskytnout i neurčité a nejasné situace, například při psaní uvozovek spolu se závorkou. V~tomto případě je problém v~ukončení věty tečkou. Tento problém dále popisuje \cite{Laurencik2018}.

\subsection{Typografie v~programování}
Typografie existuje i při psaní programů v~různých jazycích. Je důležité pochopit v~jakém kontextu se bude používat daný styl písma, píše \cite{Anonymni2009}.
\par
Jedno z~oblíbených použití je \texttt{CamelCase} styl, který se stal velmi oblíbený mezi programátory. Můžeme jej také vidět i v~názvech sowfwaru, například \texttt{VisiCalc} viz \cite{Walker2008}.


\newpage
\renewcommand{\refname}{Literatura}
\bibliographystyle{czechiso}
\bibliography{proj4}
\end{document}
