\documentclass[a4paper, 11pt]{article}
\usepackage[utf8]{inputenc}
\usepackage[left=2cm,text={17cm, 24cm},top=3cm]{geometry}
\usepackage[czech]{babel}
\usepackage[IL2]{fontenc}
\usepackage[unicode]{hyperref}
\usepackage{amssymb}
\usepackage{amsmath}
\usepackage{amsthm}
\usepackage{times}
\usepackage{multirow}
\usepackage{graphicx}
\graphicspath{ {./images/} }
\usepackage[noline,czech,ruled,longend,linesnumbered, vlined]{algorithm2e}
\usepackage{pdflscape}
\usepackage{tikz}

\title{ITY3}
\author{Simona Češková (xcesko00)}
\date{March 2021}

\begin{document}
\begin{titlepage}
\begin{center}
\Huge
\textsc{Vysoké učení technické v~Brně\\
\huge{Fakulta informačních technologií}} \\
\vspace{\stretch{0.382}}
\LARGE
Typografie a publikování -- 3. projekt \\
\Huge{Tabulky a obrázky} \\
\vspace{\stretch{0.618}}
\end{center}
{\Large \today \hfill Simona Češková (xcesko00)}
\end{titlepage}


\section{Úvodní strana}
Název práce umístěte do zlatého řezu a nezapomeňte uvést dnesní datum a vaše jméno a přijmení.

\section{Tabulky}
Pro sázení tabulek můžeme použít buď prostředí \verb!tabbing! nebo prostředí \verb!tabular!.

\subsection{Prostředí tabbing}
Při použití \verb!tabbing! vypadá tabulka následovně:


 \begin{tabbing}
\hspace{1,1in}     \= \hspace{0.45in}  \= \hspace{0.40in}    \kill
\textbf{Ovoce} \> \textbf{Cena} \> \textbf{Množství}    \\ 
Jablka    \> 25,90   \> 3 kg      \\
Hrušky    \> 27,40   \> 2,5 kg    \\
Vodní melouny   \> 35,– \> 1 kus  \\

\end{tabbing}

\noindent Toto prostředí se dá také použít pro sázení algoritmů, ovšem vhodnější je použít 
prostředí algorithm nebo \verb!algorithm2e! (viz sekce 3).

\subsection{Prostředí tabular}
Další možností, jak vytvořit tabulku, je použít prostředí \verb!tabular!. Tabulky pak 
budou vypadat takto\footnote{Kdyby byl problem s cline, zkuste se podívat třeba sem: http://www.abclinuxu.cz/tex/poradna/show/325037.}:
\\
\begin{table}[h]
    \catcode`\-=12
    \setlength{\arrayrulewidth}{0,3mm}
    \begin{center}
        \begin{tabular}{|l|c|c|} 
            \hline
            \multicolumn{1}{|c|}{} & \multicolumn{2}{c|}{\textbf{Cena}} \\
            \cline{2-3}
            {\textbf{Měna}} & \textbf{nákup} & \textbf{prodej} \\
            \hline
            EUR & 25,227 & 26,943 \\ 
            GBP & 29,368 & 31,492 \\ 
            USD & 21,260 & 22,661 \\ 
            \hline
        \end{tabular}
    \caption{Tabulka kurzů k dnešnímu dni}
    \end{center}
\end{table}


\begin{table}[h]
\begin{center}
    \catcode`\-=12
    \setlength{\arrayrulewidth}{0,3mm}
        \begin{tabular}{|c|c|}
            \hline
            A & $\neg$A \\
            \hline
            \textbf{P} & N \\
            \hline
            \textbf{O} & O \\
            \hline
            \textbf{X} & X \\
            \hline
            \textbf{N} & P \\ 
            \hline
        \end{tabular}
        \begin{tabular}{|c|c|c|c|c|c|}
            \hline
            \multicolumn{2}{|c|}{\multirow{2}{*}{A $\wedge$ B}} & \multicolumn{4}{c|}{B} \\
            \cline{3-6}
            \multicolumn{2}{|c|}{} & \textbf{P} & \textbf{O} & \textbf{X} & \textbf{N}\\
            \hline
                & \textbf{P} & P & O & X & N \\
            \cline{2-6} 
            \multirow{2}{*}{A} & \textbf{O} & O & O & N & N \\
            \cline{2-6} 
                & \textbf{X} & X & N & X & N \\
            \cline{2-6} 
            \multicolumn{1}{|l|}{} & \textbf{N} & N & N & N & N \\
            \hline
        \end{tabular}
        \begin{tabular}{|c|c|c|c|c|c|}
            \hline
            \multicolumn{2}{|c|}{\multirow{2}{*}{A $\vee$ B}} & \multicolumn{4}{c|}{B} \\
            \cline{3-6}
            \multicolumn{2}{|c|}{} & \textbf{P} & \textbf{O} & \textbf{X} & \textbf{N}\\
            \hline
                & \textbf{P} & P & P & P & P \\
            \cline{2-6} 
            \multirow{2}{*}{A} & \textbf{O} & P & O & P & O \\
            \cline{2-6} 
                & \textbf{X} & P & P & X & X \\
            \cline{2-6} 
            \multicolumn{1}{|l|}{} & \textbf{N} & P & O & X & N \\
            \hline
        \end{tabular}
        \begin{tabular}{|c|c|c|c|c|c|}
            \hline
            \multicolumn{2}{|c|}{\multirow{2}{*}{A $\rightarrow$\,B}} & \multicolumn{4}{c|}{B} \\
            \cline{3-6}
            \multicolumn{2}{|c|}{} & \textbf{P} & \textbf{O} & \textbf{X} & \textbf{N}\\
            \hline
                & \textbf{P} & P & O & X & N \\
            \cline{2-6} 
            \multirow{2}{*}{A} & \textbf{O} & P & O & P & O \\
            \cline{2-6} 
                & \textbf{X} & P & P & X & N \\
            \cline{2-6} 
            \multicolumn{1}{|l|}{} & \textbf{N} & P & P & P & P \\
            \hline
        \end{tabular}
        \caption{Protože Kleeneho trojhodnotová logika už je \uv{zastaralá}, uvádíme si zde příklad čtyřhodnotové logiky}
\end{center}
\end{table}

\pagebreak

\section{Algoritmy}
Pokud budeme chtít vysázet algoritmus, můžeme použít prostředí \verb!algorithm!\footnote{Pro nápovědu, jak zacházet s prostředím \texttt{algorithm}, můžeme zkusit tuhle stránku: \hfill \break
http://ftp.cstug.cz/pub/tex/CTAN/macros/latex/contrib/algorithms/algorithms.pdf.} nebo \verb!algorithm2e!\footnote{Pro \texttt{algorithm2e} zase tuhle: http://ftp.cstug.cz/pub/tex/CTAN/macros/latex/contrib/algorithm2e/doc/algorithm2e.pdf.}.
Příklad použití prostředí \verb!algorithm2e! viz Algoritmus 1.


\begin{algorithm}
    \SetNlSty{}{}{:}
    \SetNlSkip{-1.15em}
    \KwIn{$(X_{t-1}, u_{t}, z_{t})$}
    \KwOut{$X_t$}
    \Indp\Indpp
    $\overline{X_t} = X_t = 0$\\
    \For{$k = 1 \mathrm{\,to\,} M$}
    {
        $x_{t}^{[k]} =$ \emph{sample\_motion\_model}$(u_{t}, x_{t-1}^{[k]})$\\
        $\omega_{t}^{[k]} =$ \emph{measurement\_model}$(z_{t}, x_{t}^{[k]}, m_{t-1})$\\
        $m_{t}^{[k]} =$ \emph{updated\_occupancy\_grid}$(z_t, x_{t}^{[k]}, m_{t-1}^{[k]})$\\
        $\overline{X_{t}} = \overline{X_{t}} + \langle x_{x}^{[m]}, \omega^{[m]}_{t} \rangle$\\
    }
    \For {$k = 1 \mathrm{\,to\,} M$}
    {
        draw $i$ with probability $\approx \omega^{[i]}_{t}$\\
        add $\langle x^{[k]}_{x}, m^{[k]}_{t} \rangle$ to $X_{t}$\\
    }
    \KwRet $X_t$
    \caption{FastSLAM}
\end{algorithm}

\section{Obrázky}
Do našich článků můžeme samozřejmě vkládat obrázky. Pokud je obrázkem fotografie,
můžeme klidně použít bitmapový soubor. Pokud by to ale mělo být nějaké schéma nebo
něco podobného, je dobrým zvykem takovýto obrázek vytvořit vektorově.

\begin{figure}[h]
    \centering
    \includegraphics[width=0.25\textwidth]{etiopan.eps}
    \vspace{0mm}
    \scalebox{-1}[1]{\includegraphics[width=.25\textwidth]{etiopan.eps}}
    \caption{Malý Etiopánek a jeho bratříček}
    \label{Etiopánek}
\end{figure}

\newpage

\noindent Rozdíl mezi vektorovým \dots

\begin{figure}[h]
    \centering
    \includegraphics[width=.5\textwidth]{oniisan.eps}
    \caption{Vektorový obrázek}
    \label{vektorový obrázek}
\end{figure}

\noindent\dots a bitmapovým obrázkem

\begin{figure}[h]
    \centering
    \includegraphics[width=.5\textwidth]{oniisan2.eps}
    \caption{Bitmapový obrázek}
    \label{bitmapový obrázek}
\end{figure}


\noindent se projeví například při zvětšení
\par
Odkazy (nejen ty) na obrázky 1, 2 a 3, na  
tabulky 1 a 2 a také na algoritmus 1 jsou udělány pomocí 
křížových odkazů. Pak je ovšem potřeba zdrojový soubor přeložit dvakrát.
\par
Vektorové obrázky lze vytvořit i přímo v~LATEXu, například pomocí prostředí \verb!picture!.

\newpage
\begin{landscape}

\begin{figure}
\begin{tikzpicture}

\draw (2,2) -- (23,2) -- (23,15) -- (2,15) -- (2,2);
\draw (8,5) -- (20,5) -- (20,12) -- (8,12) -- (8,5);
\draw (20,5) -- (23,5);

\draw (5,13) circle (1cm);

\draw (8,8) -- (20,8);
\draw (8,8.7) -- (20,8.7);
\draw (8,10) -- (20,10);
\draw (8,10.7) -- (20,10.7);
\draw (9,10.7) -- (9,11.7) -- (10.5,11.7) -- (10.5,10.7);
\draw (16,10.7) -- (16,11.7) -- (14.5,11.7) -- (14.5,10.7);
\draw (9,8.7) -- (9,9.7) -- (10.5,9.7) -- (10.5,8.7);

\draw (11,10.9) -- (11,11.7) -- (13.5,11.7) -- (13.5,10.9) -- (11,10.9);
\draw (16.5,10.9) -- (16.5,11.7) -- (19,11.7) -- (19,10.9) -- (16.5,10.9);

\draw (16.5,8.9) -- (16.5,9.7) -- (19,9.7) -- (19,8.9) -- (16.5,8.9);
\draw (11,8.9) -- (11,9.7) -- (13.5,9.7) -- (13.5,8.9) -- (11,8.9);
\draw (14,8.9) -- (14,9.7) -- (16,9.7) -- (16,8.9) -- (14,8.9);

\draw (13.3,2) -- (14,4.5) -- (19,4.5) -- (19.7,2);

\draw (8,5) -- (7,2);
\draw (10,4) -- (9,2);
\draw (12,4) -- (11,2);
\draw (9,5) -- (8.7,4) -- (13.1,4) -- (13,5);

\draw (8,6) -- (6.7,2.7);
\draw (8,6.2) -- (6.6,2.78);
\draw (6.6,2.78) -- (7,2);

\draw (8,12) -- (9,14);
\draw (20,12) -- (21,14) -- (9,14);
\draw (20,5) -- (21,2);
\draw (21,14) -- (21,6) -- (20,5);
\draw (14,5) -- (14,7.5) -- (19,7.5) -- (19,5);

\draw (14,5.5) -- (19,5.5);
\draw (14,6) -- (19,6);
\draw (14,6.5) -- (19,6.5);
\draw (14,7) -- (19,7);

\draw (10,5) -- (10,7.5) -- ( 12,7.5) -- ( 12,5);
\draw (8.5,6) -- (8.5,6.5) -- (9.5,6.5) -- (9.5,6) -- (8.5,6);


\end{tikzpicture}
\caption{Vektorový obrázek mého domu.}
\end{figure}
\end{landscape}
\end{document}
